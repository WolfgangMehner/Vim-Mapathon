%%=====================================================================================
%%
%%       Filename:  talk.tex
%%
%%    Description:  
%%
%%        Version:  1.0
%%        Created:  20.09.2017
%%       Revision:  none
%%
%%         Author:  Wolfgang Mehner (WM), wolfgang-mehner@web.de
%%   Organization:  
%%      Copyright:  Copyright (c) 2017, Wolfgang Mehner
%%
%%          Notes:  
%%
%%=====================================================================================

\documentclass{beamer}

% Packages    {{{1
\usepackage[latin1]{inputenc}
\usepackage[T1]{fontenc}
\usepackage{amsmath}
\usepackage{amsfonts}
\usepackage{amssymb}
\usepackage{textcomp}

\usepackage{beamertexpower}

\usepackage{listings}

\lstset{basicstyle={\ttfamily\small}}
\lstset{commentstyle=\itshape}
\lstset{extendedchars=true}
\lstset{frame=none}
\lstset{framesep=2mm}
\lstset{identifierstyle=\ttfamily}
\lstset{keywordstyle=\bfseries}
\lstset{language=C}
\lstset{linewidth=155mm}
\lstset{numbers=left}
\lstset{numberstyle=\tiny}
\lstset{showstringspaces=false}
\lstset{stepnumber=1}
\lstset{tabsize=2}

%\usepackage[scaled]{luximono}

% Settings   {{{1
%\usetheme{Madrid}
\usetheme{Singapore}

\definecolor{darkred}{HTML}{AA0000}
\definecolor{darkgreen}{HTML}{00DD00}
\definecolor{darkblue}{HTML}{0000AA}

\definecolor{darkmagenta}{HTML}{990099}

%\makeatother
\setbeamertemplate{footline}
{%
	\leavevmode%
	\hbox{%
		\begin{beamercolorbox}[wd=.8\paperwidth,ht=2.25ex,dp=2ex,center]{author in head/foot}%
		\end{beamercolorbox}%
		\begin{beamercolorbox}[wd=.2\paperwidth,ht=2.25ex,dp=2ex,center]{author in head/foot}%
			\insertframenumber{}\hspace*{1ex}
		\end{beamercolorbox}}%
	\vskip0pt%
}

% ToC at beginning of sections
\AtBeginSection[]
{
	\begin{frame}<beamer>{Contents}
		\tableofcontents[currentsection]
	\end{frame}
}

% Code   {{{1 
\newenvironment{mycode}[0]
{\ttfamily}
{}

\definecolor{comment}{HTML}{0000A0}
\definecolor{keyword}{HTML}{800000}
\definecolor{identifier}{HTML}{2E8B57}
\definecolor{string}{HTML}{CA1F7B}

\newcommand{\myind}[1]{\textcolor{white}{#1}}
\newcommand{\mycommt}[1]{\textcolor{comment}{#1}}
\newcommand{\myident}[1]{\textcolor{identifier}{#1}}
\newcommand{\mykeywd}[1]{\textcolor{keyword}{#1}}
\newcommand{\mystring}[1]{\textcolor{string}{#1}}

\newcommand{\myctrlkey}[1]{\textcolor{blue}{#1}}

% Title   {{{1
\author[W. Mehner]{Wolfgang Mehner}
\title[Vim Maps]{Vim Mapathon}
\subtitle[Vim Maps]{An advanced introduction to maps}
\institute[]{Vimfest 2017}
\date[23.9.17]{23. September 2017}

% Abstract   {{{1
% 
% I will highlight some of the more advanced uses of maps:
% 
% - Extending maps with expressions and VimScript
% - Maps for the Vim command-line
% - Custom completion on the command-line using maps (e.g. CTRL+P style completion)
% - A short digression on custom completion for ex-commands (using tab)
% - Putting it all together to make your plug-ins (in lack of a better word) sparkle
% 
% With a bit of effort and a few recurring tricks you can match or even improve upon the comfort the command-line completion your shell offers.
% Authors of plug-ins will hopefully find some interesting suggestions, but normal users of Vim can improve their workflow as well.

% Outline   {{{1
%
% Introduction
% - Abbreviations
% - Example maps
% - Is this interesting for the regular user? (as opposed to a plug-in developer)
%   -> Yes, to a degree. Maps can hugely improve your workflow. YOUR workflow.
% - Is this interesting for the plug-in developer?
%
% Basics
% - Notes on VimScript
%   * scripting language: dynamic data structures, dynamic typing, pass-by-reference
%   * but with some unusual scoping
%   * access to options, more special variables ...
% - Filetype plug-ins
%   * for examples with buffer-local maps
% - Maps depend on modes, and might be buffer-local
% - Examples: ... (something global, something local)
% - Examples: Brackets with different modes
% - Maps and VimScript
%   * calling help
% - Expression maps with <expr>
%   * grep support
%
% Maps on the Command-Line
% - Yet another mode
% - Example: Regexp maps
% - Example: CTRL+BACKSPACE for deleting words (like shell)
% - Example: Custom completion with CTRL+P (like in insert-mode)
%
% User-Commands and Completion
% - ... a short digression, more of a developer topic ...
% - Example: target completion for make
% - Tab-completion breaks file/path completion
%
% Putting it all together:
% - Example: Proper help calls
% - Demo: Git/LaTeX/Lua?
%   * Git: menu/maps/ex-commands
%   * Git: expression maps (grep top, tagging in log buffer)
%   * Git: tab-completion (help)
%   * Git: custom completion (git log --stat luasupport-0.8..luasupport-0.9 -- plugin/lua-support.vim)

% }}}1

% Document
\begin{document}

% Front Matter   {{{1

\maketitle

% Introduction   {{{1

\section*{Introduction}

\begin{frame}{Let's start before the beginning}{}

	To get us started, \textit{Abbreviations}: \\
	\vspace{2mm}

	\begin{mycode}
		abbreviate  wtf   water-tight ferrets \\
		abbreviate  != {} \textasciitilde{}= \\
	\end{mycode}

	\vspace{2mm}

	In Insert mode, type 1st argument, obtain the 2nd
\end{frame}

\begin{frame}{Let's take a shortcut}{}

	\textit{Maps} are more general: \\
	\vspace{2mm}

	\begin{mycode}
		inoremap  (  {} {} {} {} ()\myctrlkey{<Left>} \\[2mm]
		nnoremap  <C-q>       :wqall\myctrlkey{<CR>} \\
		nnoremap  Gd {} {} {} :split\myctrlkey{<CR>}gd
	\end{mycode}

	\vspace{2mm}

	Executes the right-hand side.\\
	Can change modes, use special keys, \dots
\end{frame}

\subsection*{Goal}

\begin{frame}{Lessons to be learned}{}
	Is this interesting for the regular user? \\
	\vspace{3mm}
	\textrightarrow \hspace{1mm} Yes, maps can hugely improve your workflow. \\[1mm]
	YOUR workflow.
\end{frame}

\begin{frame}{Lessons to be learned}{}
	Is this interesting for plug-in developers? \\
	\vspace{3mm}
	\textrightarrow \hspace{1mm} Of central importance.\\
	\vspace{5mm}
	\pause
	Maps: \textcolor{darkgreen}{fast, integrate into the workflow} - \textcolor{darkred}{accessibility?} \\
	Menus: \textcolor{darkgreen}{good overview} - \textcolor{darkred}{access not fast enough} \\
	Ex-commands: \textcolor{darkgreen}{powerful and flexible} - \textcolor{darkred}{still not as fast} \\
	\vspace{2mm}
	\pause
	Combining all three: \textcolor{darkblue}{best usability and configurability}
\end{frame}

\begin{frame}{Make Vim great again?}{Already is!}

	\textcolor{darkmagenta}{How can we beat the shell \textit{for all our regular tasks}?} \\
	\vspace{2mm}
	\pause
	Provide often used commands via VimScript. \\
	And match the comfort of the shell's tab-completion! \\

	\vspace{5mm}
	\pause

	We may not beat an IDE for one specific language.\\
	\vspace{1mm}
	\textcolor{darkmagenta}{How can we provide consistent performance \textit{for a dozen languages}?} \\
	\vspace{2mm}
	\pause
	Provide accessible and configurable features! \\
	Otherwise all the power we provide is lost.

\end{frame}

\begin{frame}{Contents}{}
	\tableofcontents
\end{frame}

% Basics   {{{1

\section{Basics}

\begin{frame}{VimScript}{Vim's configuration language}
	Notes on VimScript:
	\begin{itemize}
		\item scripting language: \\
			dynamic data structures, dynamic typing, pass-by-reference \dots
		\item but with some unusual scoping: \\
			variables and functions linked to buffers, windows, scripts, \dots
	\end{itemize}
\end{frame}

\begin{frame}{VimScript}{Filetype plug-ins}
	Filetype plug-ins:
	\begin{itemize}
		\item<1-> script in VimScript, especially for one filetype
		\item<1-> executed once for each new buffer
		\item<2-> \textit{e.g.}\ define buffer-local maps
		\item<2-> thus become filetype-specific maps
	\end{itemize}
\end{frame}

\subsection{Maps and Modes}

\begin{frame}{Buffer-Local Maps}{}

	Global maps:
	\vspace{2mm}

	\begin{mycode}
		inoremap  (  ()\myctrlkey{<Left>} \\[2mm]
	\end{mycode}

	\vspace{5mm}

	Local to C++ buffers:
	\vspace{2mm}

	\begin{mycode}
		inoremap  <buffer>  \{\myctrlkey{<CR>}  \{\myctrlkey{<CR>}\}\myctrlkey{<Esc>}O
	\end{mycode} \\

	\vspace{5mm}
	Mind the \code{noremap}.

\end{frame}

\begin{frame}{Maps and Modes}{}

	Different modes warrant and require different maps:
	\vspace{2mm}

	\begin{mycode}
		inoremap  ( {} ()\myctrlkey{<Left>} \\
		vnoremap  (   s()\myctrlkey{<Esc>}P
	\end{mycode}

	\vspace{5mm}

	\begin{mycode}
		inoremap  <buffer>  \{\myctrlkey{<CR>} {} \{\myctrlkey{<CR>}\}\myctrlkey{<Esc>}O \\
		vnoremap  <buffer>  \{\myctrlkey{<CR>}   S\{\myctrlkey{<CR>}\}\myctrlkey{<Esc>}Pk=iB
	\end{mycode}

\end{frame}

\begin{frame}{Calling for Help}{}

	Open a dictionary for the word under the cursor: \\
	\vspace{2mm}
	\begin{mycode}
		nnoremap  Hen  :call CallHelpWiki()\myctrlkey{<CR>}
	\end{mycode}

	\vspace{5mm}
	\pause

	This is easiest with a bit of VimScript: \\
	\vspace{2mm}
	\begin{mycode}
		\mykeywd{function} CallHelpWiki () \\
			\myind{..}\mykeywd{let} url {} = \mystring{"https://en.wiktionary.org/wiki/\%s"} \\
			\myind{..}\mykeywd{let} word = \myident{expand} ( \mystring{"<cword>"} ) \\[2mm]

			\myind{..}\mykeywd{let} url\_f = \myident{printf} ( url, word ) \\[2mm]

			\myind{..}\mykeywd{call} \myident{system} ( \mystring{"firefox "}.url ) \\
		\mykeywd{endfunction} \\
	\end{mycode}

\end{frame}

\begin{frame}{Implementation}{}

	\begin{mycode}
		\mykeywd{function} CallHelpWiki () \\[2mm]

			\myind{..}\mykeywd{let} url {} = \mystring{"https://en.wiktionary.org/wiki/\%s"} \\
			\myind{..}\mykeywd{let} word = \myident{expand} ( \mystring{"<cword>"} ) \\
			\myind{..}\mykeywd{let} word = \myident{substitute} ( word, \mystring{'\textbackslash{}W'}, \mystring{''}, \mystring{'g'} ) \\[2mm]

			\myind{..}\mykeywd{if} word == \mystring{""} \\
			\myind{..}\myind{..}\mykeywd{echomsg} \mystring{"no word under cursor"} \\
			\myind{..}\myind{..}\mykeywd{return} \\
			\myind{..}\mykeywd{endif} \\[2mm]

			\myind{..}\mykeywd{let} url\_f = \myident{printf} ( url, word ) \\
			\myind{..}\mykeywd{call} \myident{system} ( \mystring{"firefox "}.url\_f ) \\
		\mykeywd{endfunction} \\
	\end{mycode}

\end{frame}

\begin{frame}{Expression Maps}{}

	Run Vim's \code{grep}. \\
	(In Visual mode, put the selected word on the cmd.-line.) \\
	\vspace{2mm}

	\begin{mycode}
		nmap {} {} {} {} {} {} {} <C-G>f {} {} {} {} {} {} :grep {} \%\myctrlkey{<Left><Left>} \\
		imap {} {} {} {} {} {} {} <C-G>f {}           \myctrlkey{<Esc>}:grep {} \%\myctrlkey{<Left><Left>} \\
		vmap <expr>               <C-G>f             "\myctrlkey{<Esc>}:grep ".@*." \%\myctrlkey{<Left><Left>}" \\
	\end{mycode}

	\vspace{5mm}
	\pause[2]

	\begin{overprint}
		\onslide<2>
		Maps with \code{<expr>}: \\
		Evaluate the expression, \\
		execute the result as the right-hand side: \\[2mm]
		\begin{mycode}
			\mystring{"<Esc>:grep "} .\ @* .\ \mystring{" \%<Left><Left>"}
		\end{mycode}
		\onslide<3>
		Maps with \code{<expr>}: \\
		Evaluate the expression, \\
		execute the result as the right-hand side: \\[2mm]
		\begin{mycode}
			\myind{.}\myctrlkey{<Esc>}:grep selectword \%\myctrlkey{<Left><Left>}
		\end{mycode}
	\end{overprint}

\end{frame}

% Command-Line Maps   {{{1

\section{Command-Line Maps}

\begin{frame}{Command-Line Maps}{}

	We can define maps for the command-line, using \code{cmap}:
	\vspace{2mm}

	\begin{mycode}
		cnoremap  <C-x>c  \textbackslash(\textbackslash)\myctrlkey{<Left><Left>} \\
		cnoremap  <C-x>w  \textbackslash<\textbackslash>\myctrlkey{<Left><Left>}
	\end{mycode}

	\vspace{2mm}
	In the same spirit as the brackets before.

\end{frame}

\subsection{Example: Word-wise backspace}

\begin{frame}{Let's be more ambitious}{}
	
	My shell supports \myctrlkey{<Alt-Backspace>} for deleting a whole word. \\
	On the Vim cmd.-line, use \myctrlkey{<Ctrl-W>}. \\
	\vspace{2mm}

	Let's code it ourselves as an exercise. \\

	\vspace{5mm}
	\pause

	Central trick: \\
	\vspace{2mm}
	\begin{mycode}
		cmap  LHS  \myctrlkey{<C-\textbackslash>}e\textcolor{red}{EXPR}\myctrlkey{<CR>}
	\end{mycode}

	\vspace{2mm}
	Evaluate \code{\textcolor{red}{EXPR}} and replace the command-line with the result.

\end{frame}

\begin{frame}{Implementation}{}

	Use a function CmdLineWordDelete(): \\
	\vspace{2mm}
	\begin{mycode}
		cmap  LHS  \myctrlkey{<C-\textbackslash>}e\textcolor{red}{CmdLineWordDelete()}\myctrlkey{<CR>}
	\end{mycode}

	\vspace{5mm}

	Interface:\\
	\vspace{2mm}
	\begin{mycode}
		\mykeywd{function} CmdLineWordDelete () \\
		\myind{..}\mycommt{" ...} \\
		\myind{..}\mykeywd{return} replacement\_string \\
		\mykeywd{endfunction}
	\end{mycode}

\end{frame}

\begin{frame}{Implementation}{cont.}

	\begin{overprint}

		\onslide<1>

		\begin{mycode}
			\mycommt{" current cmd.-line and position of the cursor} \\
			\mykeywd{let} cmdline = \myident{getcmdline} () \\
			\mykeywd{let} cmdpos {} = \myident{getcmdpos} () - 1 \\
			\\
			\mycommt{" split: <head><CURSOR><tail>} \\
			\mykeywd{let} cl\_head = \myident{strpart} ( cmdline, 0, cmdpos ) \\
			\mykeywd{let} cl\_tail = \myident{strpart} ( cmdline, cmdpos ) \\
			\\
			\mycommt{" replace 'cl\_head'} \\
			\mycommt{" ...} \\
			\\
			\mycommt{" set new cmdline cursor position} \\
			\mykeywd{call} \myident{setcmdpos} ( len(cl\_head)+1 ) \\
			\\
			\mykeywd{return} cl\_head.cl\_tail \\
		\end{mycode}
		
		\onslide<2>

		\begin{mycode}
			\mycommt{" ...} \\
			\\
			\mycommt{" replace} \\
			\mykeywd{if} match ( cl\_head, \mystring{'\textbackslash{}w\$'} ) > -1 \\
			\myind{..}\mykeywd{let} cl\_head = \myident{substitute}
				( cl\_head, \mystring{'\textbackslash{}w\textbackslash+\$'}, \mystring{''}, \mystring{''} ) \\
			\mykeywd{else} \\
			\myind{..}\mykeywd{let} cl\_head = \myident{substitute}
				( cl\_head, \mystring{'.\$'}, \mystring{''}, \mystring{''} ) \\
			\mykeywd{endif} \\
			\\
			\mycommt{" ...} \\
			\\
			\mykeywd{return} cl\_head.cl\_tail \\
		\end{mycode}
	\end{overprint}
\end{frame}

\subsection{Example: Keyword completion}

\begin{frame}{What else can we do?}{}
	
	Provide \myctrlkey{<Ctrl-P>} on the cmd.-line. \\
	\vspace{5mm}

	Same method as before. \\
	But we need to cycle through different keyword matches.

\end{frame}

\begin{frame}{Implementation}{}

	\begin{mycode}
		\mycommt{" current cmd.-line and position of the cursor} \\
		\mycommt{" split: <head><CURSOR><tail>} \\[2mm]

		\mykeywd{if} cl\_head \mykeywd{==} b:CmdLineLast \\
		\myind{..}\mykeywd{let} b:CmdLineIndex {} += 1 {} {} {} \mycommt{" next match} \\
		\mykeywd{else} \\
		\myind{..}\mykeywd{let} b:CmdLineMatches = ... \mycommt{" search replacements} \\
		\myind{..}\mykeywd{let} b:CmdLineIndex {} {} = 0 {} {} {} \mycommt{" first match} \\
		\mykeywd{endif} \\[2mm]

		\mycommt{" modify cl\_head} \\
		\mykeywd{let} cl\_head = ...\ .\ b:CmdLineMatches[ b:CmdLineIndex ] \\[2mm]

		\mycommt{" set new cmdline cursor position} \\[2mm]

		\mykeywd{let} b:CmdLineLast = cl\_head {} {} \mycommt{" saved for next call} \\
		\mykeywd{return} cl\_head.cl\_tail \\
	\end{mycode}
		
\end{frame}

\subsection{Proper Approach}

\begin{frame}{The proper approach}{}

	\begin{overprint}
		\onslide<1>

		Until now, public function: \\

		\begin{mycode}
			\mykeywd{function} CmdLineWordDelete () \\
			\myind{..}\mycommt{" ...} \\
			\mykeywd{endfunction} \\[2mm]
%
			cmap  <c-bs>   \myctrlkey{<C-\textbackslash>}e\textcolor{darkred}{CmdLineWordDelete()}\myctrlkey{<CR>} \\
			cmap  <c-p> {} \myctrlkey{<C-\textbackslash>}e\textcolor{darkred}{CmdLineCompletion(-1)}\myctrlkey{<CR>} \\
			cmap  <c-n> {} \myctrlkey{<C-\textbackslash>}e\textcolor{darkred}{CmdLineCompletion(1)}\myctrlkey{<CR>}
		\end{mycode}

		\onslide<2>

		Local implementation: \\

		\begin{mycode}
			\mykeywd{function} s:CmdLineWordDelete () \\
			\myind{..}\mycommt{" ...} \\
			\mykeywd{endfunction} \\[2mm]
%
			cmap  <c-bs>   \myctrlkey{<C-\textbackslash>}e\myctrlkey{<SID>}CmdLineWordDelete()\myctrlkey{<CR>} \\
			cmap  <c-p> {} \myctrlkey{<C-\textbackslash>}e\myctrlkey{<SID>}CmdLineCompletion(-1)\myctrlkey{<CR>} \\
			cmap  <c-n> {} \myctrlkey{<C-\textbackslash>}e\myctrlkey{<SID>}CmdLineCompletion(1)\myctrlkey{<CR>}
		\end{mycode}

		\onslide<3>

		Local implementation and separate configuration: \\

		\begin{mycode}
			\mykeywd{function} s:CmdLineWordDelete () \\
			\myind{..}\mycommt{" ...} \\
			\mykeywd{endfunction} \\[2mm]
%
			cmap  <Plug>MapsClWd   \myctrlkey{<C-\textbackslash>}e\myctrlkey{<SID>}CmdLineWordDelete()\myctrlkey{<CR>} \\
			cmap  <Plug>MapsClCp   \myctrlkey{<C-\textbackslash>}e\myctrlkey{<SID>}CmdLineCompletion(-1)\myctrlkey{<CR>} \\
			cmap  <Plug>MapsClCn   \myctrlkey{<C-\textbackslash>}e\myctrlkey{<SID>}CmdLineCompletion(1)\myctrlkey{<CR>}
		\end{mycode}

		\onslide<4>

		Local implementation and separate configuration: \\

		\begin{mycode}
			cmap  <c-bs>   <Plug>MapsClWd \\
			cmap  <c-p> {} <Plug>MapsClCp \\
			cmap  <c-n> {} <Plug>MapsClCn \\[2mm]
%
			cmap  <Plug>MapsClWd   \myctrlkey{<C-\textbackslash>}e\myctrlkey{<SID>}CmdLineWordDelete()\myctrlkey{<CR>} \\
			cmap  <Plug>MapsClCp   \myctrlkey{<C-\textbackslash>}e\myctrlkey{<SID>}CmdLineCompletion(-1)\myctrlkey{<CR>} \\
			cmap  <Plug>MapsClCn   \myctrlkey{<C-\textbackslash>}e\myctrlkey{<SID>}CmdLineCompletion(1)\myctrlkey{<CR>}
%
		\end{mycode}
	\end{overprint}

\end{frame}

% User-Defined Commands   {{{1

\section{User-Defined Commands}

\begin{frame}{Ex-Commands}{}

	Another powerful mechanism: \\
	User-defined \textit{ex-commands}. \\

	\vspace{5mm}
	\pause

	Consider a \code{make} plug-in: \\
	\vspace{2mm}

	\begin{mycode}
		command -complete=file MakeFile \\
		\myind{....}:call \myctrlkey{<SID>}SetMakeFile(<q-args>) \\
		command -complete=customlist,\myctrlkey{<SID>}MakeComplete Make \\
		\myind{....}:call \myctrlkey{<SID>}Run(<q-args>) \\
	\end{mycode}

	\vspace{5mm}
	\pause

	We can provide custom tab-completion.

\end{frame}

\subsection{Example: Make-target completion}

\begin{frame}{Tab-Completion}{}

	\begin{mycode}
		... -complete=customlist,\myctrlkey{<SID>}MakeComplete ... \\[5mm]

		\mykeywd{function} s:MakeComplete( ArgLead, CmdLine, CursorPos ) \\
		\myind{..}\mykeywd{let} targets = s:GetMakeTargets( s:Makefile ) \\[2mm]

		\myind{..}\mykeywd{return} \myident{filter}( \myident{copy}( targets ), ...\myident{a:ArgLead}... )) \\
		\mykeywd{endfunction}
	\end{mycode}

	\vspace{5mm}
	\pause

	But this breaks tab-completions for filenames!

\end{frame}

\begin{frame}{Tab-Completion}{}

	\begin{mycode}
		... -complete=customlist,\myctrlkey{<SID>}MakeComplete ... \\[5mm]

		\mykeywd{function} s:MakeComplete( ArgLead, CmdLine, CursorPos ) \\
			\myind{..}\mykeywd{let} files {} {} = \myident{split}( \myident{glob}( \myident{a:ArgLead}.\mystring{"*"} ), \mystring{"\textbackslash{}n"} )
			\myind{..}\mykeywd{let} targets = s:GetMakeTargets( s:Makefile ) \\[2mm]

			\myind{..}\mykeywd{return} \myident{filter}( \myident{copy}( targets ), ...\myident{a:ArgLead}...\ )\ ) \\
			\myind{.....}+ files \\
		\mykeywd{endfunction}
	\end{mycode}

	\vspace{5mm}
	\pause

	Sufficient for this use-case, \\
	but Vim's filename completion is still nicer.

\end{frame}

% Putting It All Together   {{{1

\section{Putting It All Together}

\subsection{Example: Calling help}

\begin{frame}{Calling for Help}{cont.}

	\begin{overprint}
		\onslide<1>

		Let's revisit this example: \\[2mm]
		\begin{mycode}
			\mykeywd{function} CallHelpWiki () \\[2mm]
%
			\myind{..}\mykeywd{let} url {} = \mystring{"https://en.wiktionary.org/wiki/\%s"} \\
			\myind{..}\mykeywd{let} word = \myident{expand} ( \mystring{"<cword>"} ) \\
			\myind{..}\mykeywd{let} word = \myident{substitute} ( word, \mystring{'\textbackslash{}W'}, \mystring{''}, \mystring{'g'} ) \\[2mm]
%
			\myind{..}\mykeywd{if} word == \mystring{""} \\
			\myind{..}\myind{..}\mykeywd{echomsg} \mystring{"no word under cursor"} \\
			\myind{..}\myind{..}\mykeywd{return} \\
			\myind{..}\mykeywd{endif} \\[2mm]
%
			\myind{..}\mykeywd{let} url\_f = \myident{printf} ( url, word ) \\
			\myind{..}\mykeywd{call} \myident{system} ( \mystring{"firefox "}.url\_f ) \\
			\mykeywd{endfunction} \\
		\end{mycode}

		\onslide<2>

		Use a local function and an argument: \\[2mm]
		\begin{mycode}
			\mykeywd{function} s:CallHelp ( url ) \\[2mm]
%
			\myind{..}\mykeywd{let} url {} = \myident{a:url} \\
			\myind{..}\mykeywd{let} word = \myident{expand} ( \mystring{"<cword>"} ) \\
			\myind{..}\mykeywd{let} word = \myident{substitute} ( word, \mystring{'\textbackslash{}W'}, \mystring{''}, \mystring{'g'} ) \\[2mm]
%
			\myind{..}\mykeywd{if} word == \mystring{""} \\
			\myind{..}\myind{..}\mykeywd{echomsg} \mystring{"no word under cursor"} \\
			\myind{..}\myind{..}\mykeywd{return} \\
			\myind{..}\mykeywd{endif} \\[2mm]
%
			\myind{..}\mykeywd{let} url\_f = \myident{printf} ( url, word ) \\
			\myind{..}\mykeywd{call} \myident{system} ( \mystring{"firefox "}.url\_f ) \\
			\mykeywd{endfunction} \\
		\end{mycode}

	\end{overprint}

\end{frame}

\begin{frame}{Calling for Help}{cont.}

	Provide an ex-command: \\[2mm]
	\begin{mycode}
		\mykeywd{function} s:CallHelp ( url ) \\
		\myind{..}\mycommt{" ...} \\
		\mykeywd{endfunction} \\[3mm]

		\mykeywd{command} MapathonHelp :call \myctrlkey{<SID>}CallHelp(<q-args>) \\[3mm]

		\mykeywd{nnoremap} Hen :MapathonHelp \\
		\myind{..}https://en.wiktionary.org/wiki/\textcolor{darkred}{\%s}\myctrlkey{<CR>} \\
		\mykeywd{nnoremap} Hcp :MapathonHelp \\
		\myind{..}http://en.cppreference.com/mwiki/...search=\textcolor{darkred}{\%s}\myctrlkey{<CR>} \\
		\mykeywd{nnoremap} Hqt :MapathonHelp \\
		\myind{..}http://qt-project.org/doc/qt-4.8/\textcolor{darkred}{\%s}.html\myctrlkey{<CR>}
	\end{mycode}

\end{frame}

\subsection{Demonstration}

\begin{frame}{Demonstration}{}

	Integrate some often used tools into Git. \\[5mm]

	Demonstration: Git, \dots

\end{frame}

% }}}1

\end{document}

%%----------------------------------------------------------------------
%%  vim: spell: foldmethod=marker
